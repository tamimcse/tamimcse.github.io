%%%%%%%%%%%%%%%%%
% This is an example CV created using altacv.cls (v1.1.5, 1 December 2018) written by
% LianTze Lim (liantze@gmail.com), based on the
% Cv created by BusinessInsider at http://www.businessinsider.my/a-sample-resume-for-marissa-mayer-2016-7/?r=US&IR=T
%
%% It may be distributed and/or modified under the
%% conditions of the LaTeX Project Public License, either version 1.3
%% of this license or (at your option) any later version.
%% The latest version of this license is in
%%    http://www.latex-project.org/lppl.txt
%% and version 1.3 or later is part of all distributions of LaTeX
%% version 2003/12/01 or later.
%%%%%%%%%%%%%%%%

%% If you are using \orcid or academicons
%% icons, make sure you have the academicons
%% option here, and compile with XeLaTeX
%% or LuaLaTeX.
% \documentclass[10pt,a4paper,academicons]{altacv}

%% Use the "normalphoto" option if you want a normal photo instead of cropped to a circle
% \documentclass[10pt,a4paper,normalphoto]{altacv}

\documentclass[10pt,a4paper,ragged2e]{altacv}

%% AltaCV uses the fontawesome and academicon fonts
%% and packages.
%% See texdoc.net/pkg/fontawecome and http://texdoc.net/pkg/academicons for full list of symbols. You MUST compile with XeLaTeX or LuaLaTeX if you want to use academicons.

% Change the page layout if you need to
\geometry{left=2cm,right=10cm,marginparwidth=6.8cm,marginparsep=1.2cm,top=1.25cm,bottom=1.25cm}

% Change the font if you want to, depending on whether
% you're using pdflatex or xelatex/lualatex
\ifxetexorluatex
  % If using xelatex or lualatex:
  \setmainfont{Carlito}
\else
  % If using pdflatex:
  \usepackage[utf8]{inputenc}
  \usepackage[T1]{fontenc}
  \usepackage[default]{lato}
\fi

% Change the colours if you want to
\definecolor{VividPurple}{HTML}{000000}
\definecolor{SlateGrey}{HTML}{2E2E2E}
\definecolor{LightGrey}{HTML}{2E2E2E}
\colorlet{heading}{VividPurple}
\colorlet{accent}{VividPurple}
\colorlet{emphasis}{SlateGrey}
\colorlet{body}{LightGrey}

% Change the bullets for itemize and rating marker
% for \cvskill if you want to
\renewcommand{\itemmarker}{{\small\textbullet}}
\renewcommand{\ratingmarker}{\faCircle}

%% sample.bib contains your publications
\addbibresource{sample.bib}

\begin{document}
\name{MD I\MakeLowercase{slam} (T\MakeLowercase{amim})}
%Note that, following line consumes extra space
\tagline{}
% Cropped to square from https://en.wikipedia.org/wiki/Marissa_Mayer#/media/File:Marissa_Mayer_May_2014_(cropped).jpg, CC-BY 2.0
%\photo{3.3cm}{profile.jpg}
\personalinfo{%
  % Not all of these are required!
  % You can add your own with \printinfo{symbol}{detail}
   \location{Palo Alto, CA}
  \email{mislam4@kent.edu}
 \phone{330-389-3188}
%  \mailaddress{Address, Street, 00000 County}
\homepage{https://tamimcse.github.io/}
%  \twitter{@marissamayer}
%  \linkedin{https://tamimcse.github.io/}
   \github{https://leetcode.com/tamimcse/}
%   \github{github.com/tamimcse} % I'm just making this up though.
%   \orcid{orcid.org/0000-0000-0000-0000} % Obviously making this up too. If you want to use this field (and also other academicons symbols), add "academicons" option to \documentclass{altacv}
}

%% Make the header extend all the way to the right, if you want.
\begin{fullwidth}
\makecvheader
\end{fullwidth}

%% Depending on your tastes, you may want to make fonts of itemize environments slightly smaller
\AtBeginEnvironment{itemize}{\small}

%% Provide the file name containing the sidebar contents as an optional parameter to \cvsection.
%% You can always just use \marginpar{...} if you do
%% not need to align the top of the contents to any
%% \cvsection title in the "main" bar.
% \divider
%\cvskill{German}{3}


\cvsection[page1sidebar]{Education}
\cvevent{Kent State University}{PhD in Computer Science}{ Aug 2013 -- Aug 2022}{Kent, OH}
\begin{itemize}
%\item GPA: 3.82/4.0
\item Research area: High-level Synthesis; Network Algorithms
\end{itemize}

\divider

\cvevent{BUET}{Bachelors in Computer Science and Engineering}{ Aug 2003 -- Dec 2008}{Dhaka, Bangladesh}
%\begin{itemize}
%	\item GPA: 3.45/4.0\\
%\end{itemize}

\cvsection{TECHNICAL SKILLS}
\cvtag{C++}
\cvtag{C}
\cvtag{High-Level Synthesis}
\cvtag{Network Algorithms}
\cvtag{Linux Kernel}
%\cvtag{Java}
%\cvtag{C\#}
%\cvtag{Python}
%\cvtag{Verilog}
%\cvtag{HTML/CSS}
%\cvtag{JavaScript}
\\
\cvsection{Research Interests}
\cvtag{Compilers for hardware accelerators}
\cvtag{High-Level Synthesis}
\cvtag{FPGA Physical Synthesis}
%\divider
% \cvachievement{\faTrophy}{}{Received accolades at Atos for Best Performance in team.}
% \cvachievement{\faTrophy}{}{Received Best Debut Award at Atos. }
% %\divider
% \cvachievement{\faInstitution}{}{Won 2nd Consolation Prize for paper presented on Cognitive Radio Networks.}
% %\divider
% \cvachievement{\faGraduationCap}{}{Got Selected in "Exclusive Scholar Program" during undergrad.}
% %\divider
% \cvachievement{\faDollar}{}{Awarded with Narotam Sekhsaria Foundation Scholarship}
%\cvsection{Strengths}

%\cvtag{Hard-working (18/24)} 
%\cvtag{Persuasive}
%\cvtag{Motivator \& Leader}

%\divider\smallskip

%\cvtag{UX}
%\cvtag{Mobile Devices \& Applications}
%\cvtag{Product Management \& Marketing}


%\divider

%\cvevent{B.S.\ in Symbolic Systems}{Stanford University}{Sept 1993 -- June 1997}{}

\cvsection{Projects}
\cvproject{C2RTL: A High-Level Synthesis tool}
\begin{itemize}
	\item Developed a high-level synthesis tool named C2RTL that can generate synthesizable Verilog RTL for pipelined ASIC from C code.
	\item It was designed as a GCC plugin. It takes intermediate code (produced by GCC) as an input and generates control and data-flow graph (CDFG) for that. It then performs scheduling and MUX tree generation before producing the Verilog code.
	\item Evaluated the generated Verilog code with OpenROAD 
\end{itemize}
\smallskip
\cvproject{CP-Trie: A Longest Prefix Match algorithm in Software and ASIC}
\begin{itemize}
	\item Developed several bitmap and Trie based longest prefix match algorithms such as (CP-Trie, Poptrie and SAIL) for IPv6 routing table lookup. 
	\item Evaluated the algorithms with routes from real core routers.
\end{itemize}
\smallskip
\cvproject{NC-TCP: A congestion control in Linux kernel}
\begin{itemize}
	\item Developed several router assisted congestion control (XCP, RCP and NC-TCP)  in Linux kernel.
	\item Evaluated the protocols using Mininet and a GStreamer based video streaming application. 
\end{itemize}

\cvsection{Selected Publications}
\begin{itemize}
	\item \underline{MD Iftakharul Islam}, Javed I Khan "CP-Trie: Cumulative PopCount based Trie for IPv6 Routing Table Lookup in Software and ASIC." IEEE HPSR, 2021.
	\item \underline{MD Iftakharul Islam}, Javed I Khan "C2RTL: A High-level Synthesis System for IP Lookup and Packet Classification." IEEE HPSR, 2021.
%	\item \underline{MD Iftakharul Islam}, Javed I Khan "SAIL Based FIB Lookup in a Programmable Pipeline Based Linux Router." IEEE HPSR, 2019.
%	\item \underline{MD Iftakharul Islam}, Javed I Khan "Leveraging Domino to Implement RCP in a Stateful Programmable Pipeline." IEEE HPSR, 2019.
	\item \underline{MD Iftakharul Islam}, Javed I Khan "A Network-centric TCP for Interactive Video Delivery Networks (VDN)." IEEE ICNP Workshop PVE-SDN, 2017.
	
\end{itemize}

% \divider


%\divider
%\cvsection{Languages}
%\cvskill{English}{5}
%\cvskill{Spanish}{5}
%\cvskill{Mandarin}{3}
% \cvevent{Product Engineer}{Google}{23 June 1999 -- 2001}{Palo Alto, CA}

% \begin{itemize}
% \item Joined the company as employe \#20 and female employee \#1
% \item Developed targeted advertisement in order to use user's search queries and show them related ads
% \end{itemize}

%\cvsection{A Day of My Life}

% Adapted from @Jake's answer from http://tex.stackexchange.com/a/82729/226
% \wheelchart{outer radius}{inner radius}{
% comma-separated list of value/text width/color/detail}
% Some ad-hoc tweaking to adjust the labels so that they don't overlap
% \wheelchart{1.5cm}{0.5cm}{%
%   10/10em/accent!30/Sleeping \& dreaming about work,
%   25/9em/accent!60/Public resolving issues with Yahoo!\ investors,
%   5/13em/accent!10/\footnotesize\\[1ex]New York \& San Francisco Ballet Jawbone board member,
%   20/15em/accent!40/Spending time with family,
%   5/8em/accent!20/\footnotesize Business development for Yahoo!\ after the Verizon acquisition,
%   30/9em/accent/Showing Yahoo!\ employees that their work has meaning,
%   5/8em/accent!20/Baking cupcakes
% }

\clearpage

% \cvsection[page2sidebar]{Publications}

\nocite{*}

% \printbibliography[heading=pubtype,title={\printinfo{\faBook}{Books}},type=book]

% \divider

% \printbibliography[heading=pubtype,title={\printinfo{\faFileTextO}{Journal Articles}}, type=article]

% \divider

% \printbibliography[heading=pubtype,title={\printinfo{\faGroup}{Conference Proceedings}},type=inproceedings]

% %% If the NEXT page doesn't start with a \cvsection but you'd
% %% still like to add a sidebar, then use this command on THIS
% %% page to add it. The optional argument lets you pull up the
% %% sidebar a bit so that it looks aligned with the top of the
% %% main column.
% % \addnextpagesidebar[-1ex]{page3sidebar}


\end{document}
