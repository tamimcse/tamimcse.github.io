% LaTeX file for resume 
% This file uses the resume document class (res.cls)

\documentclass{res} 
%\usepackage{helvetica} % uses helvetica postscript font (download helvetica.sty)
%\usepackage{newcent}   % uses new century schoolbook postscript font 
\newsectionwidth{0pt}  % So the text is not indented under section headings
\usepackage{fancyhdr}  % use this package to get a 2 line header
\usepackage{hyperref}
\renewcommand{\headrulewidth}{0pt} % suppress line drawn by default by fancyhdr
\setlength{\headheight}{24pt} % allow room for 2-line header
\setlength{\headsep}{24pt}  % space between header and text
\setlength{\headheight}{24pt} % allow room for 2-line header
\pagestyle{fancy}     % set pagestyle for document
\rhead{ {\it Tamim}\\{\it p. \thepage} } % put text in header (right side)
\cfoot{}                                     % the foot is empty
\topmargin=-0.5in % start text higher on the page

\begin{document}
\thispagestyle{empty} % this page has no header  
\name{ Integrating packet scheduler to WireGuard\\[12pt]}% the \\[12pt] adds a blank line after name


\begin{resume}

\section{\leftline{Objective}} 
WireGuard provides VPN tunnels through network interface peers. Unlike IPSec, the network interfaces are responsible for encrypting outgoing packets. Wireguard uses UDP between peers. Currently WireGuard has no way to combat \textit{buffer-bloat} problem. As WireGuard interfaces are responsible for encrypting outgoing data in \textit{softirq}, they should also implement the AQM techniques to avoid \textit{buffer-bloat} problem (AQM in forwarding layer is not sufficient). I would like to integrate fq\_codel and PIE qdiscs to the sending pipeline of WireGuard. I also like to optimize the qdisc implementation by parallelizing it, offloading the computation to NIC hardware and auditing locks. Finally, I would like to implement a RCP-like explicit feedback mechanism by processing L3/L4 headers (ECN, Ipv6 FlowInfo and TCP option field) of incoming data. 

\section{\leftline{Implementation}} 


\section{\leftline{Timeline}} 

Simulated a call center where calls are generated from different probabilistic distributions and assigned to agents based on their scheduled data. Different skill matrices were being calculated.

Implemented an interval tree to make the call assignment efficient instead of searching all the agents.


\section{\leftline{About Me}} 
I am a PhD student of computer science department of Kent State University. I am doing research on \textit{Network-centric TCP for interactive video streaming}. I am acquainted with TCP and Qdiscs in Linux kernel. I also worked on Mininet and GStreamer. Currently I am getting acquainted with WireGuard, Netfilter, nftable and Ethernet driver.  I had my Bachelor degree from Bangladesh University of Engineering and Technology (BUET). I worked as a software engineer for five years. I am proficient in C, C++, Java, Shell Script and Python. I deeply care about Linux and free software in general. I wish to remain engaged in Linux kernel development in the future.


My Github: \url{https://github.com/tamimcse}

My resume: \url{https://web.cs.kent.edu/~mislam4/resume.pdf}

 
 
\end{resume} 
\end{document}













